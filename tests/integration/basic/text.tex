\documentclass{article}
\usepackage[normalem]{ulem}
\usepackage{xcolor}
\begin{document}
  % A lot of this is from https://en.wikibooks.org/wiki/LaTeX/Text_Formatting

  \section{Fonts}

  This is \emph{emph} \\
  This is \textrm{textrm} \\
  This is \textsf{textsf} \\
  This is \textsc{Small Caps} \\
  This is \uppercase{uPpeRcAsE} \\
  This is \textit{textit} \\
  This is \texttt{fixed width} \\
  This is \textbf{textbf} \\
  This is \textmd{textmd} \\
  This is \textlf{textlf} \\
  This is \underline{underline} \\

  \section{ulem package}

  This is \uline{uline} \\
  This is \sout{sout} \\

  \section{Sizing}

  {\tiny tiny} \\
  {\scriptsize scriptsize} \\
  {\footnotesize footnotesize} \\
  {\small small} \\
  {\normalsize normalsize} \\
  {\large large} \\
  {\Large Large} \\
  {\LARGE LARGE} \\
  {\huge huge} \\
  {\Huge Huge} \\

  \section{Colors}

  {black text \color{red}red text}

  \colorbox{red}{\color{white}white text on red background}

  \section{Spacing}

  The last two words have a non-breaking~space.

  Before hfill. \hfill After hfill.

  \begin{doublespace}
    This paragraph has \\ double \\ line spacing.
  \end{doublespace}

  \section{Quotes}

  To `quote' in LaTeX \\
  To ``quote'' in LaTeX \\
  To ``quote" in LaTeX \\
  To ,,quote'' in LaTeX \\
  ,,German quotation marks`` \\
  <<French quotation marks>> \\
  ``Please press the `x' key.'' \\
  ,,Proszę, naciśnij klawisz <<x>>''.

  \section{Super/subscript}

  Some\textsuperscript{superscript} \\
  Some\textsubscript{subscript}

  \section{Dashes and hypens}

  Hyphen: daughter-in-law, X-rated\\
  En dash: pages 13--67\\
  Em dash: yes---or no? \\
  Minus sign: $0$, $1$ and $-1$

  \section{Ellipsis}

  Not like this ... but like this:\\
  New York, Tokyo, Budapest, \ldots
\end{document}
